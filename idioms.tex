\begin{itemize}
\item Destructuring assignments
\begin{lstlisting}
a,b,c = 1,2,3
# a is 1, b is 2 and c is 3

l = [(1,2), (3,4), (5,6)]
for i,j in l:
   print (i+j)
# Will print 3, 7, 11
\end{lstlisting}
\item Iterating with count
  \begin{lstlisting}
l = "abcdef"
for i in range(len(l)):  # Bad style
   print (i, l[i])

for i,j in enumerate(l): # Good style
   print (i, j)
  \end{lstlisting}
\item List comprehensions
  \begin{lstlisting}
l = range(1, 5)
# l is [1,2,3,4]
m = [x**2 for x in l] 
# m is [1,4,9,16]
m = [x for x in l if x%2 == 0] 
# m is [1,3] (odd numbers)
m = [x**2 for x in l if x%2]
# m is [4, 16] (squares of even nos)
  \end{lstlisting}
\item Dictionary comprehensions
  \begin{lstlisting}
d = dict (x = 2, y = 3)
d0 = {y:x for x,y in d.items()}
# d0 is {2:'x', 3:'y'}
d0 = {x:y**2 for x,y in d.items()}
# d0 is {'x':4, 'y':9}
d0 = {x:y for x,y in d.items() if y%2}
# d0 is {'x':2} (Even values only)
d0 = {x:0 for x in d}
# d0 is {'x':0, 'y':0}
  \end{lstlisting}
\item Module importing
\begin{lstlisting}
if __name__ == "__main__":
  # Will only run when module is run
  f1()
else:
  # Will only run when module is imported
  f2()  
\end{lstlisting}

This can be used to get different behaviours when importing or running
a module
\end{itemize}
