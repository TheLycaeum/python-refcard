\subsection{The REPL}
You can start the interpreter by running \texttt{python} in your
terminal. The \texttt{>>>} prompt indicates that you're inside the
interpreter. 

You can type \emph{small} programs and run them here. \keys{\ctrl + d} will
quit the interpreter.

Multiline statements (e.g. a texttt{def}), will change the to \texttt{...} to
indicating that you're inside a command. An empty line will end the
statement.

\begin{lstlisting}[numbers=left, numberstyle=\tiny\color{gray}, xleftmargin={0.75cm}]
>>> def hello(): %* \keys{\return}*)
...    print ("Hello, world") %* \keys{\return}*)
... %* \keys{\return}*)
>>> hello() %* \keys{\return}*)
Hello, world
\end{lstlisting}

On line \#1, the prompt changes to \texttt{...} when you hit
\keys{\return}. Now you're defining the function \texttt{hello}. Line
\#2 is the body. Hitting \keys{\return} on an empty line (line \#3) will
complete the function and give you the \texttt{>>>} prompt back. Now
you can run the function as in line \#4.

\subsection{Program files}
You can type a python program into a file
(e.g. \texttt{example.py}). You can import and run functions inside
them  

\begin {lstlisting}
# example.py 
def add(x, y):
   return x + y
\end{lstlisting}

\begin {lstlisting}
>>> import example # No .py. 
>>> example.add(5, 6)
11
\end{lstlisting}

To run this program directly, you can add an import guard like so and
run it.

\begin {lstlisting}[numbers=left, numberstyle=\tiny\color{gray}, xleftmargin={0.75cm}]
# example.py 
import sys

def add(x, y):
    return x + y

if __name__ == "__main__":
    a = sys.argv[1]
    b = sys.argv[2]
    print (add(int(a), int(b)))
\end{lstlisting}

Line \#2 imports the \texttt{sys} module which has command line
arguments in \texttt{sys.argv}. Line \#8 and \#9 gets the arguments
and line \#9 converts them to integers and calls \texttt{add} on them.

Line \#7 will prevent the body of the \texttt{if} from running unless
the module is directly run. It will not run when the module is imported.











