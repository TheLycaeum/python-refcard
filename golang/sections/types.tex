\subsection{Basic Types}
\begin{tabular}{l l l l l l}
\multicolumn{6}{l}{\texttt{bool}} \\
\multicolumn{6}{l}{\texttt{string}} \\
\texttt{int} & \texttt{int8} & \texttt{int16} & \texttt{int32} & \texttt{int64} \\
\texttt{uint} & \texttt{uint8} & \texttt{uint16} & \texttt{uint32} & \texttt{uint64} \\
\multicolumn{6}{l}{\texttt{uintptr}} \\
\multicolumn{6}{l}{\texttt{byte} (alias for \texttt{uint8})} \\
\multicolumn{6}{l}{\texttt{rune} (alias for \texttt{int32}) - A unicode code point} \\
\texttt{float32} & \texttt{float64} \\
\texttt{complex64} & \texttt{complex128} \\
\end{tabular}

\texttt{int}, \texttt{uint} and \texttt{uintptr} are 32 bit wide on 32-bit machines and 64 on 64-bit machines. Preferred to explict sizes.

\begin{lstlisting}
var (
  flag bool = true
  max_uint32 uint32  = 1<<32-1 
  )
\end{lstlisting}

\texttt{0} is zero value for numeric types, \texttt{false} for booleans, \texttt{""} for strings.

\texttt{T(v)} converts \texttt{v} to type \texttt{T}. No cross type assignments without conversions.

\texttt{const} can be used to define constants. e.g. \texttt{const Pi = 3.14}.

