\subsection{Arrays and slices}

\begin{lstlisting}
// Creation
var names [2]string
// Initialisation
var primes [8]{2, 3, 5, 7, 11, 13, 17, 19}

// Indexing and assignment
names[0] = "Rob"
names[1] = "Pike"
\end{lstlisting}

Cannot be resized. \emph{Size is part of the type}. A slice is a flexible view into an array.
\texttt{[]T} is a slice with elements of type \texttt{T}.

\begin{lstlisting}
  fibonacci := [6]int{1, 1, 2, 3, 5, 8}
  var n[]int = fibonacci[1:4]  //{1, 2, 3}
\end{lstlisting}

\begin{lstlisting}
  // [3]string array
  var arr = [3]string{"do", "re", "mi"}
  // []string slice
  var slice = []string{"do", "re", "mi"}
\end{lstlisting}

\emph{Length} (\texttt{len(s)}) is the number of elements in the slice.
\emph{Capacity} (\texttt{cap(s)}) is the number of in the underlying array.
\\
\texttt{make(type, len, cap)} is for dynamic arrays.
\begin{lstlisting}
  s := make([]int, 5, 15)
\end{lstlisting}
\texttt{append} will add elements to a slice. Might allocate new array.
\begin{lstlisting}
  var q []int // len(0), cap(0)
  q = append(q, 1) // len(1), cap(1)
\end{lstlisting}

Iterating over a slice can be done using \texttt{range}
\begin{lstlisting}
  for index,value := range(q) {
    fmt.Printf("%d %v\n", index, value)
    }
\end{lstlisting}
